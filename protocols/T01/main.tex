\documentclass[12pt,twoside,a4paper]{scrartcl}

\usepackage{prakstyling}
\usepackage[paper=a4paper,left=20mm,right=20mm,top=20mm,bottom=20mm]{geometry}
\usepackage{wrapfig}
\usepackage{amsmath}
\usepackage{hyperref}
\usepackage{amssymb}% http://ctan.org/pkg/amssymb
\usepackage{pifont}% http://ctan.org/pkg/pifont
%Für Literaturverzeichnis

\usepackage{biblatex}
\addbibresource{Bibliography.bib}


\newcommand{\xmark}{\ding{55}}%

%%%%%%%%%%%%%%%%%%%%%%%%%%%%%%% Autoreninfo %%%%%%%%%%%%%%%%%%%%%%%%%%%%%%%%%%%%%%%%%%%%%%%%%%
\author{Philipp Rosendahl Mat.-Nr: 378029\thanks{philipp.rosendahl@rwth-aachen.de}
		\and Lennart Wilde, Mat.-Nr: 381588\thanks{lennart.wilde@rwth-aachen.de}}

\pSetShortAuthor{378029 \& 381588}
%%%%%%%%%%%%%%%%%%%%%%%%%%%%%%%%%%%%%%%%%%%%%%%%%%%%%%%%%%%%%%%%%%%%%%%%%%%%%%%%%%%%%%%%%%%%%

%%%%%%%%%%%%%%%%%%%%%%%%%%%%%%%%%%%%%%%% TITEL %%%%%%%%%%%%%%%%%%%%%%%%%%%%%%%%%%%%%%%%%%%%%%
\pSetTitlePrefix{Experiment}
\pSetTitleNumber[T01]
\pSetLongSubject{Physikalisches Fortgeschrittenenpraktikum - Group 59} \pSetShortSubject{Group 59}
%%%%%%%%%%%%%%%%%%%%%%%%%%%%%%%%%%%%%%%%%%%%%%%%%%%%%%%%%%%%%%%%%%%%%%%%%%%%%%%%%%%%%%%%%%%%%

\setlength{\parindent}{0pt}
\pagenumbering{roman}

\raggedbottom

\renewcommand{\tablename}{Tab.}
\renewcommand{\figurename}{Fig.}
\setlength{\abovecaptionskip}{1ex}
\setlength{\belowcaptionskip}{1ex}
\setlength{\floatsep}{1ex}
\setlength{\textfloatsep}{1ex}

\begin{document}

\maketitle
\newpage

\tableofcontents
\newpage

\pagenumbering{arabic}

\section{Introduction}

	Because radiation is always present in the environment, be in in the form of cosmic rays, natural radon gas or nuclear fallout, it is of great interest to understand the properties of the different types of radiation.
	Because the different types of ionizing radiation are very different in their properties, there also exist detectors which are optimized for different types of particles. The goal of the following experiments, is to unerstand the properties of those detectors and the radiation itself better.

\section{Semiconductor Detector}

	\subsection{Setup}

		The Experiment consists of an $ ^{226}Ra$-Source, which is mounted on an x-y-z translation stage. On the tip of the source ar three holes, which form an equilateral triagnle together. With the help of two of the thee axis of the stage, the points are positioned centrally in front of the detector. For the different sections of the Experiment the data are recorded with the Genie200 Software installed on the PCs.

		\paragraph{1}
			The source is placed close to, but not directly on the detector. After that, a spectrum is recorded.

		\paragraph{2}
			The distance between the source and the detecor is increased step-by-step an a energy spectrum is recorded each time.

		\paragraph{3}
			A spectrum is recorded for the closest distance from the detector possible.

		\paragraph{4}
			The brass collimator is installed on the source and again, spectra are recorded, while moving the source away from the detector. This is coninued, until even the $^{214}Po$-Peak has disappeared.

	\subsection{Analysis}

		\paragraph{1}

			In the spectrum, you can see the different peaks of $\alpha$-Decays with different energies. It is noticeable, that because of the small energy resolution of the detector, sometimes different peaks are indistinguishable. Therefore there are only 5 Peaks clearly visible, while there are 7 Peaks expected.

			\begin{figure}
%				\includegraphics{Plots/silicon/Spectrum}
				\caption{Energy Spectrum of $ ^{226}Ra$}
				\label{SI::Spectrum}
			\end{figure}

		\paragraph{2}

		\paragraph{3}
			Using the recorded spectrum at the lowest possible distance, the channel position of each peak is computed. With the known ideal values of the energies of the corresponding particles, a linear calibration is computed by linear regression. As a result an energy calibration is obtained, which converts the channel number of the MCA to an energy value.

			\begin{align}
					E(n) &= m \cdot n + b
			\end{align}

			Where m is the energy-width of each channel and b is a constant energy offset.
			Doing the Analysis yields the following Plots and coefficients:

			\begin{figure}[H]
					\begin{minipage}{0.69 \textwidth}
						\includegraphics[width = \textwidth]{Plots/HalbleterDetektor/energyCalibrationClosest}
					\end{minipage}
					\begin{minipage}{0.29 \textwidth}
						\begin{align*}
							m &= \SI{2.2}{\kilo \electronvolt} \pm \SI{3}{\kilo \electronvolt} \\
							b &= \SI{3.8}{\mega \electronvolt} \pm \SI{2.6}{\mega \electronvolt} \\
							\frac{\chi^2}{NDF} &= 0.005
						\end{align*}
					\end{minipage}
				\end{figure}

		\paragraph{4}

			An energy calibration can also be computed for the different energies of the $^{241}Po-\alpha$ peak. By first determining the channel of the MCA with the peak a distance calibration can be obtained.

			By then using the distance calibration to calculate the expected distance those particled had left in air, when they hit the detector. Then the energy of a particle was looked up in a table provided by the NIST.
			With those information, a relation between energy and distance was established. The corresponding channel numbers and calculated energies of the peaks were the used to perform a linear regression, to obtain the calibration parameters

			\begin{figure}[H]
					\begin{minipage}{0.69 \textwidth}
						\includegraphics[width = \textwidth]{Plots/HalbleterDetektor/dis_energy_calib}
					\end{minipage}
					\begin{minipage}{0.29 \textwidth}
						\begin{align*}
							m_e &= \SI{2.3}{\kilo \electronvolt} \pm \SI{0.1}{\kilo \electronvolt} \\
							b_e &= \SI{1.6}{\mega \electronvolt} \pm \SI{0.4}{\mega \electronvolt} \\
							\frac{\chi^2}{NDF} &= 0.005
						\end{align*}
					\end{minipage}
				\end{figure}


			\paragraph{5}
				The averange range of the particles is calculated by integrating the area under the peaks for each distance.
				This results in the following:

\section{Ionization Chamber}

	\subsection{Setup}
		 The high voltage power supply of the chamber of $\approx \SI{1}{\kilo \volt}$ is switched on. The amplifier is then adjusted, so that the amperemeter reads no current flowing. After that, the $^{226}Ra$ Source is placed centrally in front of the chamber, and moved closestto the grating, without touching it. After that the distance bewteen the source and the grating is increased in steps and on each step a spectrum is recorded.

	\subsection{Analysis}

		The Experiment yielded the following raw data:

		\begin{figure}[H]
			%\includegraphics{/path/to/figure}
			\caption{Raw Data Ionization Chamber}
		\end{figure}


\section{Scintillation Counter}
	\subsection{Gamma spectrum}
	\label{Gamma}
		\subsubsection{Setup}

		There are 2 scintillator crystals (NaI(TI) and Plastic) provided. For each one of them a pulse height spectrum of a $^{137}Cs$ Source is recorded. The for each measurement Photomultiplier Tube voltage is set to $\SI{840}{\volt}$. To create the spectrum it is first necessary to amplify and shape the pulses coming from the PMT with the appropriate NIM-Modules. The output of the modules was then connected to a MCA and an oscilloscope to display the measures pulses. For each of the crystals, a pulse height spectrum of the source is recorded.

		\subsubsection{Analysis}


	\subsection{Absorption in lead}
	\ref{Absorption::Pb}
		\subsubsection{Setup}
			Using the NaI(TI) scintillator, a spectrum for different thicknesses of lead between the source and the scintillator is recorded. The PMT used is the same as in \ref{Gamma}, the voltage is again set to $\SI{840}{\volt}$.

		\subsubsection{Analysis}


	\subsection{Absorption in aluminium}
	\ref{Absorption::Al}
		\subsubsection{Setup}
				For this experiment a different PMT and schintillator was used. The voltage for the PMT was set to $\SI{900}{\volt}$ and the scintillation crystal used, was the organic scintillator anthracen. After mounting the detector assembly in the shielding, a $^{90}Sr$ was placed on the tray below it, and with the help of a counter the count rate was measured. For each absorber there were three measurements performed, so that a mean and a standard deviation could be estimated.

		\subsubsection{Analysis}

	\section{Radiation protection}

	\subsection{Measurements}
		\paragraph{1}
		The $\gamma$ dose rate near the closed safes with the radiation sources was measured from a distance of $\approx \SI{10}{\centi \metre}$.
		Also, the areas of the different setups in the lab are measured and the highest instantaneous dose rate in a time frame of 1 Minute was noted down.

		\paragraph{2}
			The areas of the different experiments in the room were examined for contamination, especially the tools to move radiation sources, counting chambers and areas behind the lead shielding. Abnormally high count rates of the $\alpha - \beta - \gamma$ detector were noted and reported to the supervisor.

	\subsubsection{Analysis}
		\paragraph{1}

			Some of the safes mentioned in the instruction could not be found, so instead the measurements were performed on the safes found in the room.
			The different measurements resulted in the following data:

			\begin{table}[H]
				\centering
				\caption{Dose rates}
				\label{Dose::safes}
				\begin{tabular}{|c|c|}
					\hline
					Safe & dose rate in $\frac{nSv}{h}$ \\
					\hline
					1		&		500 \\
					4		&		282 \\
					\hline
				\end{tabular}
			\end{table}

			The legal limit for yearly radiation doses in Germany is $\SI{1}{\milli \sievert}$ for the regular population and $\SI{20}{\milli \sievert}$ for individuals who are professionally exposed to elevated levels of radiation. A person therefore has to stay $\approx \SI{2000}{\hour}$ in close distance to the safe with the highest dose rate to be exposed to a dose of $\SI{1}{\milli \sievert}$. It is safe to say that this is very unlikely and therefore the protection of the safes frm the radiation is suficcient.

			For the dose rate near the setups nothing unusual (see \ref{Contamination}) was found, except the high dose rate above the shielding of the Mößbauer experiment, which could exceed $\SI{5}{\micro \sievert}$. This was by far the highest dose rate fount in the lab, which is not surprising, considering that the $^60Co$ source used in the experiment also has by far the highest activity of the sources used in the \textbf{Fortgeschrittenenpraktikum}.

		\paragraph{2}
		\label{Contamination}
			The chambers and tools of each workplace were checked for contamination. The results are:

			\begin{table}[H]
				\centering
				\caption{Contamination (\xmark = No, \checkmark = Yes)}
				\begin{tabular}{|c|c|c|c|c|c|c|}
					\hline
					Experiment No. & T02 & T05 & T06 & T07 & T08 & T10 \\ \hline
					Cont. Tools 	 & \xmark & \xmark & \xmark & \xmark & \xmark & \xmark \\
					Cont. chamber  & \xmark & \xmark & \xmark & \xmark & \xmark & \xmark \\ \hline
				\end{tabular}
			\end{table}

			In the end no contamination was found, but at some point a pair of pliers had a suspiciously high activity, which turned out to originate from the safe next to it.

\end{document}

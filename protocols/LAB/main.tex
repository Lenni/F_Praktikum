\documentclass[12pt,twoside,a4paper]{scrartcl}

\usepackage{prakstyling}
\usepackage[paper=a4paper,left=20mm,right=20mm,top=20mm,bottom=20mm]{geometry}
\usepackage{wrapfig}
\usepackage{amsmath}
\usepackage{hyperref}
%Für Literaturverzeichnis

\usepackage{biblatex}
\addbibresource{Bibliography.bib}



%%%%%%%%%%%%%%%%%%%%%%%%%%%%%%% Autoreninfo %%%%%%%%%%%%%%%%%%%%%%%%%%%%%%%%%%%%%%%%%%%%%%%%%%
\author{Philipp Rosendahl Mat.-Nr: 378092\thanks{philipp.rosendahl@rwth-aachen.de}
		\and Lennart Wilde, Mat.-Nr: 381588\thanks{lennart.wilde@rwth-aachen.de}}

\pSetShortAuthor{378092 \& 381588}
%%%%%%%%%%%%%%%%%%%%%%%%%%%%%%%%%%%%%%%%%%%%%%%%%%%%%%%%%%%%%%%%%%%%%%%%%%%%%%%%%%%%%%%%%%%%%

%%%%%%%%%%%%%%%%%%%%%%%%%%%%%%%%%%%%%%%% TITEL %%%%%%%%%%%%%%%%%%%%%%%%%%%%%%%%%%%%%%%%%%%%%%
\pSetTitlePrefix{Versuch}
\pSetTitleNumber[LAB]
\pSetLongSubject{Physikalisches Fortgeschrittenenpraktikum - Gruppe 59} \pSetShortSubject{Gruppe 59}
%%%%%%%%%%%%%%%%%%%%%%%%%%%%%%%%%%%%%%%%%%%%%%%%%%%%%%%%%%%%%%%%%%%%%%%%%%%%%%%%%%%%%%%%%%%%%

\setlength{\parindent}{0pt}
\pagenumbering{roman}

\raggedbottom

\renewcommand{\tablename}{Tab.}
\renewcommand{\figurename}{Fig.}
\setlength{\abovecaptionskip}{1ex}
\setlength{\belowcaptionskip}{1ex}
\setlength{\floatsep}{1ex}
\setlength{\textfloatsep}{1ex}

\begin{document}

\maketitle
\newpage

\tableofcontents
\newpage

\pagenumbering{arabic}

\section{Einleitung}
  LabView ist eine grafische Programmiersprache, mit der Wissenschaftler Front- und Backends für Virtuelle Messinstrumente programmieren können.
	In den einzelnen Versuchen wurden für die Messungen selbst erstellte oder zur Verfügung gestellte LabView Programme verwendet. In den einzelnen Experimenten wurden verschiedene einfache Elektronische Schaltungen oder Bauteile vermessen, oder ihre Funktion analysiert.

	\section{Experimente}
		\subsection{Charakterisierung eines Widerstandes}

			\subsubsection{Aufbau und Durchführung}
				Mithilfe eines Rastersteckbrettes wurde die skizzerte Schaltung aufgebaut. Dabei wird der durch den Messwiderstand fließende Strom durchn den Spannungsabfall an einem Rferenzwiderstand bestimmt. Dieser Referenzwiderstand wird vorher mit einem Multimeter auf seinen Widerstandswert geprüft, und in die Auswertung mit einbezogen.

				\begin{figure}[H]
					\centering
					\fbox{
					\includegraphics[width = 0.5 \textwidth]{Pictures/Platzhalter.jpg}
					}
					\caption{Schaltplan Steckbrett}
				\end{figure}

				Dann wurde auf dem PC mit der LabView Software ``Widerstandsmessung'' die Messung gestartet und anschließend die aufgenommenen Daten gespeichert.

				Diese Prozedur wurde dann mit 3 weiteren Widerständen wiederholt, wobei zu diesen jeweils ein passender Referenzwiderstand gewählt wurde.
				Es ist außerdem darauf zu achten dass die Gesamtstromaufnahme der Schaltungen nie mehr als $\SI{10}{\milli \ampere}$ beträgt, da ansonsten die Spannungsversorgung der LabView Karte zusammenbricht.

			\subsubsection{Auswertung}

				Die Aufgenommenen Daten können aus dem Anhang \ref{Daten::Widerstand} entnommen werden.

				Die Exportierten Daten stellen die die am zu vermessenden Widerstand abfallende Spannung, abhängig vom durch ihn fließenden Strom dar. Dabei ist zu beachten dass der Strom nicht direkt gemessen, sondern indirekt über den Spannungsabfallan einem Referenzwiderstand gemessen wurde. An diesen Datenpunkten kann nun eine lineare Regression durchgeführt werden, um anhand ihrer Steigung den Widerstand des Bauteil zu Charakterisieren. Dabei stellt die Modellfunktion eine Abhängigkeit der Werte in der Form:
				\begin{align*}
					U(I) = R \cdot I + b
				\end{align*}
				dar. Dabei kann man neben der $\chi^2$-Verteilung auch den Parameter b verwenden um die Qualität des Fits und der Daten zu beurteilen. Dieser sollte sehr klein sein, da er einen intrinsischen, konstanten Spannungsabfall beschreibt, wie er z.B. in einer Diode, nicht aber in einem Widerstand auftreten sollte.

				Führt man die lineare Regression durch, so erhält man folgendes Ergebnis:

				\begin{figure}[H]
					\centering
					\fbox{
						\includegraphics[width = 0.7 \textwidth]{Pictures/Platzhalter} % {Plots/LinReg_Widerstand}
					}
					\caption{Lineare Regression für die Messdaten}
			\end{figure}

				\input{Data/LinReg_Resistance.tex}
				Als Referenzwiderstand wurde hier der Widerstand $R_R = \SI{000}{\ohm}$ verwendet, wobei dieser durch den an ihm auftretenden Spannungsabfall die Messung leicht verfälscht.

				Diese Auswertung wurde mit den 3 anderen vermessenen Widerständen mit dem gleichen Prinzip, jedoch anderen Referenzwiderständen durchgeführt.

				Dabei wurden folgende Ergebnisse ermittelt:

					\begin{table}[H]
  \centering
  \caption{Messergebnisse und Vergleich mit Erwartungswerten}

  \begin{tabular}{|c|c|c|c|}
    No. & gemessener Wert & erwarteter Wert & $\sigma$-Umgebung \\
    \hline
    1 & & &  \\
    2 & & &  \\
    3 & & &  \\
    4 & & &  \\
  \end{tabular}
\end{table}


				Man kann sehen dass \textbf{LabView für eine solche Messung vollkommen ungeeignet ist.}

			\subsubsection{Fazit}

				\textbf{TODO}

		\subsection{Kondensator}
			\subsubsection{Ziel}
				In diesem Versuchsabschnitt soll mittels des Auf- und Entladevorgangs eines Kondensators über einen bekannten Widerstand die Zeitkonstante $\tau$ bestimmt werden.

			\subsubsection{Aufbau und Durchführung}

				Die Schaltung wird wie in der Skizze gezeigt auf dem Steckbrett aufgebaut. Dabei wurde für den Kondensator $C_1$ eine Kapazität von $C_1 = \SI{000}{\micro \farad}$ und für den Widerstand ein Wert von $\SI{000}{\ohm}$ gewählt, um eine große Zeitkonstante zu erhalten. Im Anschluss muss das Programm für die automatische Aufnahme der Lade- und Entladekurven in LabView progammiert werden.(siehe Anhang \ref{Programme::Kondensator}) Mit dem fertigen Programm werden im Anschluss die Messdaten aufgezeichnet. Dabei wurden folgende Einstellungen verwendet:

				\begin{itemize}
					\item Samplerate:
					\item Gesamtsamples:
					\item Ladespannung:
				\end{itemize}

			\subsubsection{Auswertung}

				An die aufgenommenen Messdaten (siehe \ref{Daten::Kondensator}) kann nun für die Entladekurve eine Funktion der Form
				\begin{align*}
					U(t) = U_0 \cdot e^{-\frac{\tau}{t}}
				\end{align*}
				,und die Ladekurve eine Funktion der Form
				\begin{align*}
					U(t) = U_0 \cdot \qty(1-e^{-\frac{\tau}{t}})
				\end{align*}
				angepasst werden. Dabei stellt $U_0$ die zuvor im Programm eingestellte Ladespannung, sowie $\tau$ die zu bestimmende Zeitkonstante dar.

				Die jeweiligen Fehler auf die einzelnen Datenpunkte setzen sich durch den Fehler auf die Spannungsmessung, welche durch die Spannungsauflösung der Messkarte gegeben ist, sowie der Zeitauflösung, durch die Abtastrate des Programms dar. In unseren Fall belaufen sich diese Aufbau::Kammer

				\begin{align}
					\label{Kondensator::Messfehler}
					\Delta U &= \frac{\SI{10}{\volt}}{\sqrt{12} \cdot 2^{N}} = \SI{000}{\milli \volt}
					\Delta t &= \frac{1}{\SI{000}{\per \second}} =  \SI{000}{\milli \second} %%Abtastrate einfügen
				\end{align}

				Damit erhält man nun folgende Fits und Parameter:

				\begin{figure}[H]
					\centering
					\begin{minipage}{0.4 \textwidth}
						\includegraphics[width = 0.9\textwidth]{Pictures/Platzhalter.jpg}
						\caption{Anpassung Entladekurve}
					\end{minipage}
					\begin{minipage}{0.4 \textwidth}
							\includegraphics[width = 0.9\textwidth]{Pictures/Platzhalter.jpg}
							\caption{Anpassung Ladekurve}
					\end{minipage}
				\end{figure}

				\begin{table}
					\centering
					\begin{minipage}{0.4 \textwidth}
						\input{Data/CapDischarge}
						\caption{parameter Entladekurve}
					\end{minipage}
					\begin{minipage}{0.4 \textwidth}
						\input{Data/CapCharge}
						\caption{Parameter Ladekurve}
					\end{minipage}
				\end{table}

			\subsubsection{Fazit}
				Wenn man diese gemessenen Werte mit den erwarteten für diese Wahl von Widerstand und Kondensator vergleicht, sieht man in \ref{Kondensator::Vergleich}, dass unsere Werte in einer $???? \sigma$-Umgebung liegen. \textbf{Da ich das Protokoll hier vorschreibe habe ich allerdings noch keine Ahnung was genau das heißt.}

	\subsection{Spannungsstabilisierung mit der Z-Diode}

		\subsubsection{Aufbau und Durchführung}

			Die Schaltung wird wie in Grafik \ref{Aufbau::Z} aufgebau und mit der LabView karte verkabelt. Dann wird zur Messdatenaufnahme das entsprechende LabView VI\footnote{siehe Anhang \ref{Programme::Z}} ausgeführt. Dabei wird einmal der durch den Stromkreis fließende Strom und die an der Zener-Diode abfallende Spannung aufgezeichnet. Die Messungen werden mit 3 unterschiedlichen Vorwiederständen $R_V \in \{ \SI{47}{\ohm}, \SI{100}{\ohm}, \SI{1000}{\ohm} \}$

			\begin{figure}[H]
				\centering

				\includegraphics[width = 0.8 \textwidth]{Pictures/Platzhalter}
				\label{Aufbau::Z}
				\caption{Aufbau Spannungsstabilisierung}
			\end{figure}

			\subsubsection{Auswertung}
				Die Zener-Diode kann aufgrund ihrer niedrigen Durchbruchspannung dazu verwendet werden um Spannungsquellen bein schwankender Stromaufnahme oder Versorgungsspannung diese zu stabilisieren. Dabei wird aus einem Vorwiderstand und der Diode in Sperrrichrtung ein Spannungsteiler aufgebaut, bei dem die über der Z-Diode abfallende Spannung die stabilisierte Versorgungsspannung darstellt. Da die Durchbruchspannung allerdings trotzdem noch von dme fließenden Strom durch die Diode abhängt ergibt sich bei einer Änderung der Eingangsspannung auch eine Änderung der Ausgangsspannung. Diese wird durch den sog. relativen Glättfaktor spezifiziert, der in diesem Versuch für die Diode bestimmt werden soll.

				Der Faktor wird bestimmt, indem an einen gewählten Arbeitspunkt eine Tangente angepasst wird, deren Steigung im Idealfall der gesuchte Faktor und Achsenabschnitt die Durchbruchspannung der Z-Diode ist.
				Diese Auswertung wird für alle 3 Vorwiederstände wiederholt und läuft immer gleich ab, weswegen hier nur eine beispielhafte Auswertung gezeigt wird.

				\begin{figure}[H]
					\begin{minipage}{0.49 \textwidth}
						\includegraphics[width = 0.9 \textwidth]{Pictures/Platzhalter}
					\end{minipage}
					\begin{minipage}{0.49 \textwidth}
						\begin{align*}
							m &= 0 \pm 5.7 \\
							b &= \SI{130.5}{ \volt } \pm \SI{0.35}{\volt} \\
							\frac{\chi^2}{NDF} &= 1.6
						\end{align*}
					\end{minipage}
				\end{figure}

			\subsubsection{Fazit}

	\subsection{Gleichrichter}

		\subsubsection{Aufbau und Durchführung}
			Die Schaltung wird wie in der Skizze gezeigt aufgebaut und die Messung mit dem LabView VI Speicheroszilloskop durchgeführt. Als Wechselspannungsquelle wird der zur Verfügung stehende Signalgenerator verwendet. Die Signalparameter können aus der Versuchsanleitung entnommen werden.

		\subsection{Auswertung}

			Die eingehende Sinusspannungwird durch den Gleichrichter in eine gepulste Gleichspannung umgewandelt, wie man in Abbildung \ref{Gleichrichter::Daten} sehen kann.

			\begin{figure}
				\centering

				\includegraphics[width = 0.8 \textwidth]{Pictures/Platzhalter}

				\caption{Aufgenommene Daten}
				\label{Gleichrichter::Daten}
			\end{figure}

			Die Amplitude der Sinusspannung wird durch das Anfitten einer Sinusfunktion der Form:

			\begin{align*}
				U(t) = U_0 \cdot \sin(\omega t + \varphi)
			\end{align*}

			bestimmt. Der Gleichspannungswert kann durch das Ablesen der Spitzenwerte der gleichgerichteten Spannungswerte ermittelt werden.


			\begin{figure}[H]
				\begin{minipage}{0.49 \textwidth}
					\includegraphics[width = 0.9 \textwidth]{Pictures/Platzhalter}
				\end{minipage}
				\begin{minipage}{0.49 \textwidth}
					\begin{align*}
						m &= \SI{1269.3}{} \pm \SI{5.7}{} \\
						b &= \SI{130.5}{ \volt } \pm \SI{0.35}{\volt} \\
						\frac{\chi^2}{NDF} &= 1.6
					\end{align*}
				\end{minipage}
			\end{figure}


	\subsection{Transistor}
		\subsubsection{Aufbau und Durchführung}
			Die Schaltung wird wie in der Skizze gezeigt aufgebaut und die Messung mit dem LabView VI Speicheroszilloskop durchgeführt.

		\subsubsection{Auswertung}



\section{Anhang}
	\subsection{Rohdaten}
		\subsubsection{Charakterisierung Widerstand}
		\label{Daten::Widerstand}

			\begin{figure}[H]
				\centering
				\begin{minipage}{0.4 \textwidth}
					\fbox{
						\includegraphics[width = \textwidth]{Pictures/Platzhalter}
					}
				\caption{Fit Widerstand 1 ($\SI{000}{\ohm}$)}
				\end{minipage}
				\begin{minipage}{0.4 \textwidth}
					\fbox{
						\includegraphics[width = \textwidth]{Pictures/Platzhalter}
					}
				\caption{Fit Widerstand 2 ($\SI{000}{\ohm}$)}
				\end{minipage}
			\end{figure}

			\begin{figure}[H]
				\centering
				\begin{minipage}{0.4 \textwidth}
					\fbox{
						\includegraphics[width = \textwidth]{Pictures/Platzhalter}
					}
				\caption{Fit Widerstand 3 ($\SI{000}{\ohm}$)}
				\end{minipage}
				\begin{minipage}{0.4 \textwidth}
					\fbox{
						\includegraphics[width = \textwidth]{Pictures/Platzhalter}
					}
				\caption{Fit Widerstand 4 ($\SI{000}{\ohm}$)}
				\end{minipage}
		\end{figure}

		\subsubsection{Vermessung Zeitkonstante}
		\label{Daten::Kondensator}
			\begin{figure}[H]
				\centering
				\begin{minipage}{0.4 \textwidth}
					\fbox{
						\includegraphics[width = \textwidth]{Pictures/Platzhalter}
					}
				\caption{Fit Entladekurve}
				\end{minipage}
				\begin{minipage}{0.4 \textwidth}
					\fbox{
						\includegraphics[width = \textwidth]{Pictures/Platzhalter}
					}
				\caption{Fit Ladekurve}
				\end{minipage}
			\end{figure}

	\subsection{Programme}
		\subsubsection{Widerstandsvermesssung}

		\subsubsection{Zeitkonstante}
			\label{Programme::Kondensator
			}
		\subsubsection{Schmitt Trigger}

		\subsubsection{Das andere halt}
\end{document}
